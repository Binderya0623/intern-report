%----------------------Нүүр хуудастай хамаатай зүйлс----------------------------
\pagenumbering{roman}
\makefrontpage
\maketitle

\doublespace

% Decleration
\begin{huge}
\textbf{Зохиогчийн баталгаа}
\end{huge} \\ \ \\ 
\doublespace
Миний бие \@author \ нь "\@title" \ сэдэвтэй дадлагын ажлыг гүйцэтгэсэн болохыг зарлаж дараах зүйлсийг баталж байна:
\begin{itemize}
\item Энэхүү ажлын аль нэг хэсгийг эсвэл бүхлээр нь ямар нэг их, дээд сургуулийн зэрэг горилохоор оруулаагүй болно.
\item Бусдын хийсэн ажлаас хуулбарлаагүй, ашигласан бол ишлэл, зүүлт хийсэн.
\item Ажлыг би өөрөө (хамтарч) хийсэн ба миний хийсэн ажил, үзүүлсэн дэмжлэгийг дадлагын ажилд тодорхой тусгасан. 
\item Ажилд тусалсан бүх эх сурвалжид талархаж байна.  
\end{itemize} 
\ 

Гарын үсэг: \underline{\hspace{5cm}}

Огноо: 	\ \ \underline{\hspace{3cm}}
\newpage
%----------------------------------------------------------------------------------------
\begin{figure}
	\centering
	\includegraphics[width=17cm]{images/todorhoilolt.png}
	\label{fig:todorhoilolt}
\end{figure}
%----------------------------------------------------------------------------------------
\newpage
%----------------------------------------------------------------------------------------
% Хавсралтын нэр. Хавсралт гэдэг үг агуулахгүй
\begin{figure}
	\centering
	\includegraphics[width=17cm]{images/unelgee.png}
	\label{fig:unelgee}
\end{figure}
%----------------------------------------------------------------------------------------
\newpage
%----------------------------------------------------------------------------------------
\begin{figure}
	\centering
	\includegraphics[width=17cm]{images/plan.png}
	\label{fig:plan}
\end{figure}
%----------------------------------------------------------------------------------------
%   Агуулгын хэсэг
% \begin{table}[h]
% \caption{Дадлагын төлөвлөгөө}
% \begin{tabular}{|p{0.5cm}|p{8cm}|l|l|p{3cm}|}
% \hline
% \text{№} & \text{Гүйцэтгэх ажил} & \text{Хугацаа} & \text{Биелэлт} & Удирдагчийн үнэлгээ \\ \hline
% 1 & Програм хангамжийн үлгэр загваруудын тухай судлах & 08/11 - 08/15 & дууссан & \\ \hline
% 2 & Жава, Spring Boot, Spring MVC технологиудын тухай судлах & 08/11 - 08/15 & дууссан & \\ \hline
% 3 & Auftragsverwaltung системийн классын диаграмыг гаргах & 08/12 - 08/14 & дууссан & \\ \hline
% 4 & Auftragsverwaltung систем дээр ашигласан зохиомжийн үлгэр загваруудыг олж тогтоох & 08/14 - 08/18 & дууссан & \\ \hline
% 5 & МУИС-ын дипломын ажлыг удирдах системийн шаардлагатай танилцах  & 08/18 & дууссан & \\ \hline
% 6 & Уг шаардлагын дагуу системийн зохиомжийг загварчлах & 08/18 - 08/20 & дууссан & \\ \hline
% 7 & Системийн хэрэгжүүлэлтийг Жава технологи ашиглан гүйцэтгэх & 08/20 - 08/22 & дуусаагүй & \\ \hline
% 8 & Банкны Excel хуулгыг боловсруулах модулийн шинжилгээг гүйцэтгэх & 08/22 & дууссан & \\ \hline
% 9 & Уг шаардлага дээрээ үндэслэн зохиомж болон архитектурыг гаргах & 08/23 - 08/24 & дууссан & \\ \hline
% 10 & Уг зохиомжийн дагуу модулийн хэрэгжүүлэлтийг Spring Boot фреймворк ашиглан гүйцэтгэх & 08/25 - 08/29 & дууссан & \\ \hline
% 11 & Модулийг JUnit санг ашиглан тестлэх & 08/27 - 08/30 & дууссан & \\ \hline
% \end{tabular}
% \end{table}
%----------------------------------------------------------------------------------------

% Гарчгийг автоматаар оруулна
\setcounter{tocdepth}{1}
\tableofcontents

% Зургийн жагсаалтыг автоматаар оруулна
\listoffigures

% Хүснэгтийн жагсаалтыг автоматаар оруулна
\listoftables

% Кодын жагсаалтыг автоматаар оруулна
\lstlistoflistings

\newpage
%% \addtocontents{lof}{Зураг~\hfill Хуудас \par}
\newpage
%% \addtocontents{lot}{Хүснэгт~\hfill Хуудас \par}

\renewcommand{\cftlabel}{Зураг}


\doublespace
\pagenumbering{arabic}