\begin{abstract}
 \quad \quad \quad Миний бие \@author \ нь үйлдвэрлэлийн дадлагын 3 долоо хоногийн хугацаанд программ хангамжийн зохиомжийн үлгэр загваруудын хүрээнд урвуу инженерчлэл -ийн аргачлал, Жава, Spring Boot, Spring MVC гэсэн технологиуд дээр голчлон ажилласан ба уг технологиуд ямар шалтгаанаар үүссэн, хөгжүүлэлтийн ямар арга барил ашигладаг, компаниуд хэрхэн үүн дээр хөгжүүлэлт хийж эцсийн бүтээгдэхүүнийг гаргадаг талаар судлах -ын тулд Java Spring Boot фрэймворк ашигладаг компани болох "Монгол Ай Ди" байгууллагыг сонгон авч мэргэжлийн дадлагаа гүйцэтгэлээ.

  \quad \quad \quad Дадлагын эхний долоо хоногт би бараа захиалгийг зохицуулах "Auftragsverwaltung" систем дээр урвуу инженерчлэлийн аргачлал ашиглан зохиомжийн үлгэр загваруудыг уг системд хэрхэн ашигласан байгааг олж тогтоосон. Үүний дараа МУИС-ийн дипломын ажлыг удирдах системийн шаардлагийн дагуу системийн зохиомжийг гаргаж, уг зохиомж дээр үндэс -лэн системийг Жава технологи ашиглан хэрэгжүүллээ.

  \quad \quad \quad Дадлагын сүүлийн долоо хоногт Монгол Ай Ди компанийн хөгжүүлж буй MinuPOS системийн асуудал шийдвэрлэх хэсэгт ажилласан. Миний хариуцаж авсан модуль нь хэрэглэг- чийн дансны Excel хуулгыг сервер рүү ачаалж, хуулга доторх өгөгдлийг боловсруулан систем- ийн өгөгдлийн сантай уялдуулж, хэрэглэгчийн зээлийн мэдээллийг шинэчлэх үүрэгтэй. Уг модуль дээр ажиллахын тулд би Spring Boot, Spring MVC технологийн талаар судалгаа хийж, компанийн хөгжүүлэлтийн арга барилтай танилцаж, өгөгдсөн асуудлыг шийдвэрлэлээ.
  
  % \quad \quad \textbf{Зорилго} React болон Next.js технологийн талаар судалж, компанийн хөгжүүлэлтийн арга барилтай танилцах
  
  % \quad \quad \textbf{Зорилт} Удирдагчийн зааварчилгааны дагуу алхам алхмаар судалгаа хийж өгсөн шаардлагын хүрээнд хэрэгжүүлэлт хийх
\end{abstract}
