\section{Жава технологи}
Жава нь 1995 онд Sun Microsystems компаниас гарсан, объект хандалтат программчлалын хэл бөгөөд өнөөдөр дэлхий даяар хамгийн өргөн хэрэглэгддэг программчлалын хэлүүдийн нэг юм. Жава технологийг ашиглан вэб, десктоп, мобайл болон сервер талын программ хангамжийг хөгжүүлэх боломжтой. Жава технологийн гол давуу талууд нь платформ хооронд ажиллах чадвар (Write Once, Run Anywhere), хүчирхэг стандарт сангууд, аюулгүй байдал, олон хэрэглэг- чийн дэмжлэг зэрэг юм. 

\section{Spring Boot}
Spring Boot нь Жава дээр суурилсан, вэб болон enterprise application хөгжүүлэхэд зориулагд- сан фреймворк юм. Spring Boot ашигласнаар Spring-ийн үндсэн тохиргоонуудыг автоматаар хийж өгдөг тул хөгжүүлэгчид богино хугацаанд, цэвэр код бичиж, хурдан хөгжүүлэлт хийх боломжтой. Spring Boot нь REST API, MVC архитектур, security, data access зэрэг олон чухал боломжуудыг дэмждэг. Жишээ нь, энгийн REST API endpoint дараах байдлаар үүсгэж болно:

\begin{lstlisting}[language=Java, caption=Spring Boot REST Controller, frame=single]
@RestController
public class HelloController {
	@GetMapping("/hello")
	public String hello() {
		return "Hello from Spring Boot!";
	}
}
\end{lstlisting}

\section{Tomcat вэб сервер}
Apache Tomcat нь Java Servlet болон JSP (JavaServer Pages) технологиудыг дэмждэг, нээлт- тэй эхийн вэб сервер юм. Tomcat нь вэб программыг ажиллуулах, HTTP request-үүдийг хүлээн авах, боловсруулах үүрэгтэй. Spring Boot болон бусад Java вэб framework-ууд Tomcat-ийг embedded байдлаар ашиглах боломжтой тул deployment хийхэд хялбар болдог. Spring Boot application нь Tomcat embedded байдлаар ажилладаг бөгөөд дараах байдлаар main class-ыг үүсгэнэ:

\begin{lstlisting}[language=Java, caption=Spring Boot main class, frame=single]
@SpringBootApplication
public class Application {
	public static void main(String[] args) {
		SpringApplication.run(Application.class, args);
	}
}
\end{lstlisting}

\section{JPA (Java Persistence API)}
JPA нь Java-д өгөгдлийн сангийн persistent буюу байнгын хадгалалт хийх стандарт API юм. JPA-г ашигласнаар өгөгдлийн сангийн хүснэгтүүдийг объект хэлбэрээр удирдах, CRUD (Create, Read, Update, Delete) үйлдлүүдийг хялбар гүйцэтгэх боломжтой. JPA нь Hibernate, EclipseLink зэрэг олон төрлийн implementation-уудтай бөгөөд Spring Data JPA нь эдгээрийг Spring framework-тэй хослуулан, өгөгдлийн сангийн үйлдлийг автоматжуулж, хөгжүүлэлтийн хурдыг нэмэгдүүлдэг.

JPA ашигласнаар SQL query бичих шаардлага багасч, өгөгдлийн сангийн бүтэц өөрчлөг- дөхөд кодын засвар хийхэд хялбар болдог. Мөн энтити классуудыг annotation ашиглан тодор- хойлж, өгөгдлийн сангийн хүснэгтүүдтэй холбох боломжтой. Repository interface-ууд нь стандарт CRUD үйлдлүүдийг автоматаар дэмждэг тул хөгжүүлэгчид илүү цэвэр, богино код бичих боломжтой.

Жишээ нь, энтити класс болон repository дараах байдлаар үүсгэнэ:

\begin{lstlisting}[language=Java, caption=JPA Entity, frame=single]
@Entity
public class User {
	@Id
	@GeneratedValue(strategy = GenerationType.IDENTITY)
	private Long id;
	private String name;
	// getter, setter
}
\end{lstlisting}

\begin{lstlisting}[language=Java, caption=Spring Data JPA Repository, frame=single]
public interface UserRepository extends JpaRepository<User, Long> {
}
\end{lstlisting}

Spring Data JPA ашигласнаар өгөгдлийн сангийн хүснэгтүүдтэй ажиллах, хайлт хийх, өгөгдөл хадгалах зэрэг үйлдлүүдийг маш энгийн байдлаар гүйцэтгэх боломжтой болдог. Ингэс- нээр бизнес логик дээр төвлөрч, хөгжүүлэлтийн үр бүтээмжийг нэмэгдүүлдэг.
